SQL Assignment
Schema :
Emp :- empid, empname, dob, doj, salary, deptid, add1,add2, cityid
Dept:- deptid, deptname,
City:- cityid, cityname
1) Write SQL query to print total employee in the organization
   select count(empid) from emp;

2) Write SQL query to count total employee in different department
	select d.deptname,e.* from Dept d join (select count(empid),deptid from Emp group by deptid) e using (deptid);
	
3) Write SQL query to count total employee from different cities
    select city.cityname,city.cityid,count(empname)FROM city inner join emp on emp.cityid = city.cityid group by city.cityid, cityname order by cityname;

4) Write SQL query to print all record of employee who’s salary is greater than 3rd
minimum salary and less than 3
rd maximum salary.
	select * from emp where salary between(select salary from emp group by salary order by salary asc limit 2,1) order by salary; and
	(select salary from emp group by salary order by salary desc limit 2,1) order by salary;

5) Write SQL to print all record of employee who have working from last 15 years.
	select *, date_format(from_days(datediff(now(), doj)), '%Y') as age from emp where date_format(from_days(datediff(now(), doj)), '%Y')>= 15;

6) Write SQL to print all record of employee who have joined between two given dates.
	select  * from emp where doj between '2020-01-01' AND '2022-12-31';

7) Write SQL to print all record of employee who have joined in last one year
	select *, date_format(from_days(datediff(now(), doj)), '%Y') as age from emp where date_format(from_days(datediff(now(), doj)), '%Y')< 1;

8) Write SQL to print all record of employee who has birthday on 31,24 and 6 of day.
	select *,day(dob) as day from emp where day(dob) in(31,21,6);


9) Write SQL to print all record of employee who has age more than 45 year.
     select *, date_format(from_days(datediff(now(), dob)), '%Y') as Age from emp where date_format(from_days(datediff(now(), dob)), '%Y')>= 30;

10) Write SQL to update salary by 15% of all employee who has working more than 5
years but less than 10 years.
	update emp set salary = salary + (salary * 15/100) where date_format(from_days(datediff(now(), doj)), '%Y')>5 and date_format(from_days(datediff(now(), doj)), '%Y')<10;

---------------------------------------------------------------------------------------------
Schema:
User:
Userid,username, password, dob, date_of_registration, cityid,
Hobbies:- hobyid, hob_name.
User_hobbies:- uhid, userid, hobyid
Category:- catid, catname
Post;- postid, posttitle, content, userid, date_time, catid
1) Write SQL query to print user details with maximum number of post.
	select u.*,p.postid from user as u JOIN post as p  on u.userid = p.userid where u.userid=p.userid 
	group by u.username  having count(p.userid) = (select max(count)   
	from (select count(p.userid) as count FROM user AS u,post AS p   
	where u.userid=p.userid  group by u.username ) AS c ) order by u.username; 

2) Write SQL query to print user details with minimum number of post.
	select u.*,p.postid from user as u JOIN post as p  on u.userid = p.userid where u.userid=p.userid 
	group by u.username  having count(p.userid) = (select min(count)   
	from (select count(p.userid) as count FROM user AS u,post AS p   
	where u.userid=p.userid  group by u.username ) AS c ) order by u.username; 

3) Write SQL query to print user details with hobbies less than 3.
select * from user u join user_hobbies uh on u.userid = uh.userid join hobbies h on h.hobyid = uh.hobyid group by u.username having count(*) <3;

4) Write SQL query to print user details with no hobbies.
 select * from user u join user_hobbies uh on u.userid = uh.userid join hobbies h on h.hobyid = uh.hobyid group by u.username having count(*) =0;

5) Write SQL query to print user details with only 1 hobbies.
select * from user u join user_hobbies uh on u.userid = uh.useridjoin hobbies h on h.hobyid = uh.hobyid group by u.username having count(*) =1;

6) Write SQL query to print user details who has not published any record in last 7, 14 or
30 days.
select * from user u join post p on p.userid = u.userid where p.date_time not in (CURRENT_DATE() - interval 7 day,CURRENT_DATE() - interval 14 day,CURRENT_DATE() - interval 30 day);

7) Write SQL query to print user details who has published more than 2 post a day.
select * from user u join post p on p.userid = u.userid group by p.date_time having count(*) >=2

8) Write SQL query to print category name with most posts.
	select * from usercategory u join post p on p.userid = u.catid group by p.catid having count(*) >1;


9) Write SQL query to print category name with least posts.
	select * from usercategory u join post p on p.userid = u.catid group by p.catid having count(*) =1;

10) Write SQL query to print category name which has no post.
	select * from usercategory where catid not in(select catid from post);

